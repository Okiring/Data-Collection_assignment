\documentclass[12pt,letterpaper]{article}
\begin{document}

\title{COLLEGE OF COMPUTING AND INFORMATION SCIENCES\\ DEPARTMENT OF COMPUTER SCIENCE\\ SCHOOL OF COMPUTING AND INFORMATICS TECHNOLOGY\\}
\maketitle

\title{NAME:	         OKIRING PAUL \space-214016765  - 14/U/13973/EVE   
\maketitle
\\
\\
\begin{center}
\title {\textbf{CONCEPT}}
\end{center}
\title{\textbf{IS MONEY A MAJOR NECESSITY IN A RELATIONSHIP AMONG THE YOUTH}}
\maketitle

\section{INTRODUCTION}
Relationships (or even finding the right person to have a relationship with) can be pretty complicated. To make it work, you have to have something in common, you need to be able to communicate, and there must be some special chemistry going on. A romantic might tell you that love and money shouldn’t have anything to do with each other, but while money won’t make a relationship, it can break them. This issue has been on for a long time on whether money really matters in a relationship among the youth. Is it the no.1 most priority? Is it the only thing that can sustain a relationship? We are yet to find out the answer

\section{PROBLEM STATEMENT}
Money is so strong to the extent it can break that special bond between you and your love partner. Some relationships go south (break) when it comes to money matters. Finding out whether money is a top priority will help make awareness and reduce on rampant breakups.  Collecting this data will help us find out whether money really is a major necessity. This will in return help to improve on the health of the various relationships happening out there.

\section{TYPE OF DATA TO COLLECT}
I will collect first hand data i.e. primary data.  Primary data is basically first hand information collected and compiled. It’s the most original data that has not undergone any sort of statistical treatment. 

\section{METHODS TO USE}
\subsection{Personal investigation}
This method basically involves that researcher conducting the survey him/herself and collecting the data from it. Data collected in this way is usually accurate and reliable. 

\subsection{Through telephone}
This method involves acquiring information through telephone. It’s also a quick method and gives accurate information. I will use applications like WhatsApp, facebook, messenger etc to communicate to various people.

\subsection{Interviewing}
                Semi structured Interview guide will be used to collect data from the informants because of the following reasons
                \begin{enumerate}
  
    \item The method offers high response quality\\
    \item It takes advantage of the facilitators’ presence \\
    \item combines questioning, cross-examination and probing approaches \\
    
    \end{enumerate}
    
 \section{POPULATION}
 The intended population is the youth within Kampala district. These will be youth aged 18 and above. Having an equal number of boys and girls, individuals under the age of 18 will be considered in my study since they don’t meet the requirements. The sample size is 30 participants. Individuals will be recruited using random sampling. Participants will be recruited at various public places.
               
                
\section{SETTING}
Wandegeya will be the first location am going to acquire some of my participants from. Reason being there lots of students available, some within hostels and others at university halls, Most of which are above the age of 18.  I will also try out other locations like bugolobi, nakawa, mutungo and kitintale.


\end{document} 